\documentclass[a4paper,12pt,notitlepage]{article}

\usepackage{pgfgantt}

\usepackage[cm]{fullpage}
\setlength{\parindent}{0pt} 

\begin{document}

\title{Project Plan}
\author{Tom Jones\\u04tj8}
\maketitle

\section{Introduction}
Generated content is the corner stone of long lifed games. Wether procedurally or
user generated the ability for the game to be different everytime makes players 
want to return again and again.

By using generated worlds, we can present to the player an entire new world to 
explore and adventure in each time. Since the earliest days of computer gaming
generated worlds have been used, nethack \cite{nethack}
is one of the longest running software
projects ever, and the idea of generated worlds continues today in games like 
minecraft \cite{minecraft} and dwarf fortress \cite{dwarffortress}.\\

\noindent
In my final year project I want to explore the area of generated worlds, with a 
focus around generating interesting and appealing landscapes.

\section{Goals}
\subsection*{Rendering}
Generating interesting worlds is worthless if there is no way to visualise their
structure. My first goal is to create a system that can render all the necessary
base components required in an interesting world.

\subsection*{Generation}
Generation will take two parts, concurrently with the creation of the rendering 
system basic generation techniques will be created along with a modular engine
for retreival and creation of chunks.

\subsection*{More Advandced Rendering}
Moving forward from a world made of only terrain features will require advances
in the renderer. Support for voxel meshes and textures will allow the display of 
more immersive worlds. Other advanced rendering techniques will also be used to
improve on the worlds quality, Dynamic lighting effects and realistic water 
features.

\subsection*{Improved Generation}
The first generation goal is to generate terrain and landscape features, more 
advanced techniques can be used to improve on these worlds. Inclusion of forests,
rivers, villages will make world much more interesting to navigate. Techniques 
such as biomes will also be used to improve the generation.

\section{Methodology}
This project will mostly be an exploratory programming exercise. The final 
project will be an on the fly generated demo of the world. Something similar to
a flight simulator will be the final product.\\

An issue and milstone tracking tool will be used during development and the major
milestones roughly in order are, Basic Rendering then Generation, Advanced 
Rendering and Generation. There are many sub components within these which are
exposed in the timeline.

\section{Required Resources}
For my project I am planning to use the objective c programming language and the
OpenGl ES api for interacting with graphics hardware. The best setup for 
development would be an Apple computer running Mac OS X (10.7) and the xcode 
development environment. A graphics card capable of handling modest tasks is also
required. 

It would be desirable to run the simulation on machine with greater capacity than
my apple based laptop. The intent is to access the department of music Mac labs
to take advantage of their more powerful machines.

\section{Risk Assessment}
\subsection*{Risk: Project Runs over}
This project is not an all or nothing situation. If time allocations are not
sufficent to complete all goals or if the author becomes bogged down, it will
be possible to still produce interesting results.\\

To ensure the output at the end of the project is at the highest standard, it has
been structured in a way that allows the project to always be in a 'good' state.

\subsection*{Risk: Project encounters technical limitations}
Due to the nature of the project it is likely that some sort of technological 
bottle neck is reached. To alieviate this the project will have a modular 
structure to make it as simple as possible to evaluate components.

\section{Timetable}

\begin{ganttchart}[vgrid, hgrid, time slot modifier=0,x unit=0.8cm]{15}
	\gantttitle{Project Plan}{15} \\
	
	\gantttitle{Jan}{1} 
	\gantttitle{Feb}{3} 
	\gantttitle{Mar}{4} 
	\gantttitle{Apr}{5} 
	\gantttitle{May}{2} \\

	\ganttbar[bar/.style={fill=green!50},
			bar label font={\small\MakeUppercase}]
					{Basic Rendering}{0}{4} \\
	\ganttbar[bar/.style={fill=red!50},
			bar label font={\small\MakeUppercase}]
					{Data Structures}{1}{3} \\
	\ganttbar[bar/.style={fill=blue!50},
			bar label font={\small\MakeUppercase}]
					{Basic Generation}{2}{7} \\
	\ganttbar[bar/.style={fill=yellow!50},
			bar label font={\small\MakeUppercase}]
					{Data Management}{0.5}{1.5} \\
	\ganttbar[bar/.style={fill=green!50},
			bar label font={\small\MakeUppercase}]
					{Advanced Rendering}{7}{11} \\
	\ganttbar[bar/.style={fill=blue!50},
			bar label font={\small\MakeUppercase}]
					{Advanced Generation}{10}{13} \\
\end{ganttchart}


\section{References}
\begin{thebibliography}{9}


\bibitem{nethack}
  Leslie Lamport,
  \emph{\LaTeX: A Document Preparation System}.
  Addison Wesley, Massachusetts,
  2nd Edition,
  1994.
\bibitem{minecraft}
  Leslie Lamport,
  \emph{\LaTeX: A Document Preparation System}.
  Addison Wesley, Massachusetts,
  2nd Edition,
  1994.
\bibitem{dwarffortress}
  Leslie Lamport,
  \emph{\LaTeX: A Document Preparation System}.
  Addison Wesley, Massachusetts,
  2nd Edition,
  1994.
\end{thebibliography}
\end{document}
