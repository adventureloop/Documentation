\section{Generation Techniques}

\subsection{Simulated Generation}
Simulating the formation of the universe and planets is a very active area of 
research. Simulating the formation of the world is a technique that is rarely
used, but can created very realistic worlds.\\

Theres are a number of problems with
simulation the entire generation of a world that put it beyond our abilities. 
The first issue is one of time, simulations can be very resource intensive and 
can take a very long time to finish, our goal of creating the world on the fly 
puts this outside our reach. The next issue is that of scale, simulations needs
limits for each generation, or the simulation will just run for ever. With other
techniques we can create infinite world, this is much more desirable. These two
problems are enough to rule out creating generated worlds using simulation.\\

\subsection{Random Generation}
Taking values from a random or pseudo random number generate is a common 
technique to provide a random element within a system. Random Values can lead to
systems with a high level of entropy and there are a number of algorithms and 
techniques that take advantage of this method to create terrain simulations. Such
techniques are the main mechanic in the game Nethack, used to generate winding
dungeons that are different very time.\\

\subsubsection*{Mid Point Displacement}
The midpoint displacement algorithm works around raising relative levels around a
point, an offset grid is the best medium for this to work in. This leads to noisy
variation, it can be good for providing variaton in a flat or rolling landscape.\\

MDP IMAGE\\

MDP works around picking a random point within the offset grid and increasing its
height slightly compared to those around. This can be done in a large number of 
interations and the height variance can be altered with each, the final product 
being very noisy. Every cycle of this algoritms has a large cost and it is very
difficult to simualte edge conditions for unloaded chunks. MPD can be effect, but
expensive to run in increasing generations.\\

MDP RESULT IMAGE\\

\subsubsection*{Fault Algorithm}
The fault algorithm can implement either comparative raising or both raising and
lowering or height levels. Terrain generated with fault algorithms can be 
inherentyl smoother than MDP, yet takes a large number of generations to produce
interesting terrain.\\

FAULT IMAGE\\

The Fault algoritm works on a 2D grid, a line( or fault) is then drawn across the
grid, heights on either side of this line are raised and lowered respectivly.
Smoothing can be added to improve the quality of the faulting, a sigmoidal 
function produces quite a realistic value. The Fault algorithm shares the same
draw backs as the MDP algorithm.\\

FAULT RESULT IMAGE\\

\subsection{Noise Based Generation}
During the development of the ground breaking CGI film Tron, Ken Perlin found 
that the generated scenes in the film were too 'clean', they need a form of noise
applied to them to appear more life like and not purely computer generated. 
Perlin Noise was created to alieviate that problem, A multidimensional functions
that provides a value in the [-1,1] range for a give position ( (x,y) in 2D 
noise). Perline Noise has proven to be very use in creating realistic blended 
textures and can be used to generate height maps for realistic terrain and is a 
commonly used technique to this day for improving the realism of a rendered or
generated world.\\

\subsubsection*{Perlin Noise}
The basic idea around Noise generation is to take values at a given interval from
a psuedorandom number generator, and interpolate between the two, effectivly 
smoothing between the generated intervals. Perlin Noise can be generated in any
dimension, but for this document we will use 1D noise as an example and 2 and 
3 dimensional noise for generation. This document uses the original C function
libraries written by Ken Perlin for his noise generation.\\ 

\begin{lstlisting}
double PerlinNoise1D(double x,double alpha,double beta,int n);
double PerlinNoise2D(double x,double y,double alpha,double beta,int n);
double PerlinNoise3D(double x,double y,double z,double a,double b,int n);
\end{lstlisting}

For the above dimensions it can be seen that the noise function takes in the 
position for the desired value, in the number of dimensions. N is usually given
in the 6 to 10 range. The alpha value effects how noisy the final values are,with
1 being the noisest and increasing values producing smoother functions.\\

1D NOISE IMAGES\\


\subsection{Conclusion}
The desired properties of the final generation system are the best criteria to
evaluate the possible approached. We aim to create a system that can generate
the terrain under the users position on the fly, this generation needs to be
reproducable despite the order in which the user interacts with the world. Most
of all the approach used needs to be a fast as possible to avoid locking up of
the system.\\

\begin{description}
\item[Simulation]
Simulated worlds can create highly realistic
and expansive terrain, but it suffers on speed, taking a long time to create a 
world. This speed factor rules out the possibility for use in an on the fly 
generated world such as we are aiming to create. 

\item[Random Generation] Random Generation, has 
interesting qualities, but is slow to create interesting worlds and it best
suited to fixed sizes. It is possible to use an overall general map, but 
difficult to reproduce out of ordered generations which we are likely to
experience. 

\item[Perlin Noise] Noise based generation is fast, atleast compared to the other
methods, and is entirely reproducable from any state. These two qualities make it
perfect for our requirements. Noise can also be used on multiple dimensions, 
allowing height maps to be generated as well as 3D density maps leading to very
interesting realistic worlds.
\end{description}

Considering our requirements noised based generation is the best overall 
approach,
but as more complex worlds are create inspiraion will come from other 
techniques.\\
