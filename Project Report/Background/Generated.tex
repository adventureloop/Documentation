\section{Generation Techniques}

\subsection{Simulated Generation}
By simulating the world we can generate stuff


\subsection{Random Generation}
Taking values from a random or pseudo random number generate is a common 
technique to provide a random element within a system. Random Values can lead to
systems with a high level of entropy and there are a number of algorithms and 
techniques that take advantage of this method to create terrain simulations. Such
techniques are the main mechanic in the game Nethack, used to generate winding
dungeons that are different very time.\\

\subsubsection*{Mid Point Displacement}
The midpoint displacement algorithm works around raising relative levels around a
point, an offset grid is the best medium for this to work in. This leads to noisy
variation, it can be good for providing variaton in a flat or rolling landscape.\\

MDP IMAGE\\

MDP works around picking a random point within the offset grid and increasing its
height slightly compared to those around. This can be done in a large number of 
interations and the height variance can be altered with each, the final product 
being very noisy. Every cycle of this algoritms has a large cost and it is very
difficult to simualte edge conditions for unloaded chunks. MPD can be effect, but
expensive to run in increasing generations.\\

MDP RESULT IMAGE\\

\subsubsection*{Fault Algorithm}
The fault algorithm can implement either comparative raising or both raising and
lowering or height levels. Terrain generated with fault algorithms can be 
inherentyl smoother than MDP, yet takes a large number of generations to produce
interesting terrain.\\

FAULT IMAGE\\

The Fault algoritm works on a 2D grid, a line( or fault) is then drawn across the
grid, heights on either side of this line are raised and lowered respectivly.
Smoothing can be added to improve the quality of the faulting, a sigmoidal 
function produces quite a realistic value. The Fault algorithm shares the same
draw backs as the MDP algorithm.\\

FAULT RESULT IMAGE\\

\subsection{Noise Based Generation}
During the development of the ground breaking CGI film Tron, Ken Perlin found 
that the generated scenes in the film were too 'clean', they need a form of noise
applied to them to appear more life like and not purely computer generated. 
Perlin Noise was created to alieviate that problem, A multidimensional functions
that provides a value in the [-1,1] range for a give position ( (x,y) in 2D 
noise). Perline Noise has proven to be very use in creating realistic blended 
textures and can be used to generate height maps for realistic terrain and is a 
commonly used technique to this day for improving the realism of a rendered or
generated world.\\

\subsubsection*{Perlin Noise}
The basic idea around Noise generation is to take values at a given interval from
a psuedorandom number generator, and interpolate between the two, effectivly 
smoothing between the generated intervals. Perlin Noise can be generated in any
dimension, but for this document we will use 1D noise as an example and 2 and 
3 dimensional noise for generation. This document uses the original C function
libraries written by Ken Perlin for his noise generation.\\ 

\begin{lstlisting}
double PerlinNoise1D(double x,double alpha,double beta,int n);
double PerlinNoise2D(double x,double y,double alpha,double beta,int n);
double PerlinNoise3D(double x,double y,double z,double a,double b,int n);
\end{lstlisting}

For the above dimensions it can be seen that the noise function takes in the 
position for the desired value, in the number of dimensions. N is usually given
in the 6 to 10 range. The alpha value effects how noisy the final values are,with
1 being the noisest and increasing values producing smoother functions.\\
