\newpage

\section{Generation Techniques}

\subsection{Simulated Generation}
Simulating the formation of the universe and planets is a very active area of 
research. Simulating the formation of the world is a technique that is rarely
used, but can created very realistic worlds.\\

The techniques used by dwarf fortress for simulating world terrain are a good
example of common methods. DFs first pass is to create contents by using fractal
division, this allows large scale island and world features that have a self 
simial shape. DF then does a number of passes to simulate errosion, weather 
patterns, prevailing wind flow and other world simulation methods.\\

Theres are a number of problems with
simulation the entire generation of a world that put it beyond our abilities. 
The first issue is one of time, simulations can be very resource intensive and 
can take a very long time to finish, our goal of creating the world on the fly 
puts this outside our reach. The next issue is that of scale, simulations needs
limits for each generation, or the simulation will just run for ever. With other
techniques we can create infinite world, this is much more desirable. These two
problems are enough to rule out creating generated worlds using simulation.\\

\subsection{Random Generation}
Taking values from a random or pseudo random number generate is a common 
technique to provide a random element within a system. Random Values can lead to
systems with a high level of entropy and there are a number of algorithms and 
techniques that take advantage of this method to create terrain simulations. Such
techniques are the main mechanic in the game Nethack, used to generate winding
dungeons that are different very time.\\

\subsubsection*{Mid Point Displacement}
Midpoint displacement, or the diamond squared algorithm works on a 2D grid and uses
multiple passes which vary the terrain levels in each to generate its landscape.\\

\begin{figure}[h!]
\begin{center}$
\begin{array}{cc}
\includegraphics[width=0.25\textwidth]{Images/mpd1.jpg}
\includegraphics[width=0.25\textwidth]{Images/mpd2.jpg}
\end{array}$
\end{center}
\caption{MPD Grid}
\end{figure}

The grid is can initialised to random values, a preset or even zeros this depends on
the desired result for the terrain. The first step is to find a value for the central
point E. Where the value from the random function RAND(d) is within the range,[-d,d].
D reprsents the maximum displacement in the current iteration.\\

\begin{math}
E = (A+B+C+D) / 4 + RAND(d)\\
\end{math}

What follows is the square step. midpoints between the four corners, are computed as
an average of the connected points plus a random displacement.\\

\begin{math}
F = (A + C + E + E) / 4 + RAND(d)\\
G = (A + B + E + E) / 4 + RAND(d)\\
H = (B + D + E + E) / 4 + RAND(d)\\
I = (C + D + E + E) / 4 + RANS(d)\\
\end{math}

Once the first iteration has been completed it is necessary to multiply the value of
\emph{d} as shown. Where \emph{r} is the roughness constant for the terrain being
generated.\\

\begin{math}
d *= pow(2,-r)\\
\end{math}

Now we have four squares the process is repeated for each of the next squares. First
each square applies the diamond step, calculates the height for each of the center 
points. Then the square point is applied to all of the diamonds to create then next
level of values.\\

\begin{figure}[h!]
\begin{center}$
\begin{array}{cc}
\includegraphics[width=0.25\textwidth]{Images/mpd3.jpg}
\includegraphics[width=0.25\textwidth]{Images/mpd4.jpg}
\includegraphics[width=0.25\textwidth]{Images/mpd5.jpg}
\end{array}$
\end{center}
\end{figure}

\subsubsection*{Fault Algorithm}

\begin{wrapfigure}{r}{0.14\textwidth}
	\vspace{-40pt}
	\begin{center}
		\includegraphics[width=0.25\textwidth]{Images/grid.png}
	\end{center}
\end{wrapfigure}
The fault algorithm works on a grid similar to the MDP algorithm. The grid again can
be initialised to random or preset values. Fault is a very simple algorithm and with
a large number of iterations can create very interesting terrain. \\

The processing stage of the algorithm is to create a fault on the grid, a line 
bisecting the grid. The next stage is to raise and lower the height either side of
this grid, the method used to raise and lower the grid is where the algorithm can be 
varied. The simpliest technique for this is a linear motion, but it is much more 
effective to use a sigmoidal or similar function to perform this change.\\

\begin{figure}[h!]
	\begin{center}$
		\begin{array}{cc}
			\includegraphics[width=0.35\textwidth]{Images/3DTechTerrain1.png}
			\includegraphics[width=0.35\textwidth]{Images/3DTechTerrain16.png}
			\includegraphics[width=0.35\textwidth]{Images/3DTechTerrain100.png}
		\end{array}$
	\end{center}
	\caption{1,16 and 100 Iterations}
\end{figure}

\subsection{Noise Based Generation}
Noise functions produce random values in the real range [-1,1] based of some 
predefined function, purely random noise is very noisy and produces no noticeable
pattern. The noise we are going to explore for generation is called Coherent
Noise, that is instead of just random values changing across the pattern there 
is a smooth scaling between transitions within the noise.\\

\begin{figure}[h!]
	\begin{center}$
		\begin{array}{cc}
			\includegraphics[width=0.35\textwidth]{Images/noise.png}
			\includegraphics[width=0.35\textwidth]{Images/perlin.png}
		\end{array}$
	\end{center}
	\caption{Random Noise versus Perline Noise}
\end{figure}
\newpage

\subsubsection*{Perlin Noise}
The type of noise we are going to use for generation is called Perlin Noise. 
Perlin Noise was developed for use in the movie Tron by Ken Perlin, he wanted to
break up the un realistically clean lines and shapes that the graphics system was
producing. Perlin Noise is very frequently used for generation in games and is 
even built into some programming languages.\\

Perlin Noise can be generated in by number of dimensions, but we will look at the
2D case to describe an approximation of how we derive the value for a given 
(x,y).\\
\begin{wrapfigure}{l}{0.25\textwidth}
	\begin{center}
			\includegraphics[width=0.25\textwidth]{Images/cornergridpoints.png}
	\end{center}
\end{wrapfigure}

We define our noise function on a regular grid, where each grid point is defiened as
a whole number and any fractional point lies within a grid. If we want to find the
value of noise(x,y), where x and y and both not whole numbers we have a point that
lies somewhere within a square on our grid. The first thing we do is to look at the
four points which surround our desired point, these are (x0,y0),(x0,y1),(x1,y0),
(x1,y1).\\

\begin{wrapfigure}{r}{0.30\textwidth}
	\vspace{-30pt}
	\begin{center}
			\includegraphics[width=0.28\textwidth]{Images/gradients.png}
	\end{center}
\end{wrapfigure}

If we use the function g(xgrid,ygrid) = (gx,gy), which creates a pseudo random 
gradient for each of our grid points. This function needs to give an appearance of
randomness, but is an important step for the reproducability of the noise function
on multiple runs.\\

\begin{wrapfigure}{l}{0.30\textwidth}
	\vspace{-25pt}
	\begin{center}
		\includegraphics[width=0.30\textwidth]{Images/cornervectors.png}
	\end{center}
\end{wrapfigure}
For each grid point we also need to create the vector the vector from the grid point
to the point we are trying to find, this is easily calculated by subtracting the 
grid point from (x,y)\\

These gradient are now enough to start calculating the noise value for our function,
the first step is to calculate the influence of each random gradient on the final
output of the noise function and to generate weighted averages as our output.\\

The influence of each gradient is calculated by calculating the dot product of the
gradient and the vectors that go to our point, the (x,y).\\
\begin{math}
s = g(x0, y0) · ((x, y) - (x0, y0)) ,\\ 
t = g(x1, y0) · ((x, y) - (x1, y0)) ,\\
u = g(x0, y1) · ((x, y) - (x0, y1)) ,\\
v = g(x1, y1) · ((x, y) - (x1, y1)) .\\
\end{math}

If we project into 3D space and create a diagram that best explains the directions of
the vectors we have calculated.\\
\begin{wrapfigure}{r}{0.30\textwidth}
	\begin{center}
		\includegraphics[width=0.25\textwidth]{Images/zcomponents.png}
	\end{center}
	\caption{Calculated Vectors}
\end{wrapfigure}

It is important to note that our four values (s,t,u,v) are uniquely determined by
our grid point (x,y) and the random gradient function we have used. These values need
to be combined to create our final output from the noise(x,y) function. To do this
we take a weighted average of the four values, first s and t, then u and v using the
final average of these to computed values. We use a weighted average, because when
our point is closer to a grid value we want it to take more influence from that grid
value.\\

\begin{wrapfigure}{l}{0.255\textwidth}
	\vspace{-40pt}
	\hspace{-20pt}
	\begin{center}
		\includegraphics[width=0.25\textwidth]{Images/scurve.png}
	\end{center}
\end{wrapfigure}

We use an ease curve to help exaggerate the input proximity to either one or zero.
This curve means that values close to zero will be really close to zero and those
close to one will be really close to one, values that fall close to half way will be
exactly half way. Using the ease curve in this manner allows us to create the smooth
noise that we want to use.\\

To calculate our value we take the value of the ease curve at x-x0 to get a weight Sx
.Then we can find an average of s and t bu contructing a linear function that maps 0
to s and 1 to t, and evalutating it at our x dimension weight Sx. This forms the 
average a,we do the same again for u and v calling the result b.\\

\begin{figure}[h!]
\begin{center}$
\begin{array}{cc}
\includegraphics[width=0.27\textwidth]{Images/calcsx.png}
\includegraphics[width=0.25\textwidth]{Images/calca.png}
\includegraphics[width=0.25\textwidth]{Images/calcb.png}
\end{array}$
\end{center}
\caption{Find the value for Sx}
\end{figure}


\begin{figure}[h!]
\begin{center}$
\begin{array}{cc}
\includegraphics[width=0.26\textwidth]{Images/calcsy.png}
\includegraphics[width=0.25\textwidth]{Images/answer.png}
\end{array}$
\end{center}
\caption{Find Sy curve and map final value}
\end{figure}

To find our y dimension weight,Sy we evaluate the curve y-y0 and take a weighted sum
of a and b to create our final noise value z.\\

\subsubsection*{Using Perlin Noise}
Generating Perlin Noise if quite involved and there are further optimisations that
can be added to improve the algorithm. To Save on development time it makes sense to
use a library instead of reimplementing these noise functions again. The chosen 
library is the C version that Ken Perlin provides along side his SIGRAPH talk on 
generating noise. Due to its small size and use of standard C it is a very easy 
library to integrate.\\

\begin{lstlisting}
double PerlinNoise2D(double x,double y,double alpha,double beta,int n);
\end{lstlisting}

The interface to the library is very simple as can be seen from the above source
listing.
For the two dimension it can be the noise function takes in the 
position for the desired value, the number of dimensions alpha, beta and an N value.
N is usually given
in the 6 to 10 range. The alpha value effects how noisy the final values are,with
1 being the noisiest and increasing values producing smoother functions.\\

\subsection{Conclusion}
The desired properties of the final generation system are the best criteria to
evaluate the possible approached. We aim to create a system that can generate
the terrain under the users position on the fly, this generation needs to be
reproducible despite the order in which the user interacts with the world. Most
of all the approach used needs to be a fast as possible to avoid locking up of
the system.\\

\begin{description}
\item[Simulation]
Simulated worlds can create highly realistic
and expansive terrain, but it suffers on speed, taking a long time to create a 
world. This speed factor rules out the possibility for use in an on the fly 
generated world such as we are aiming to create. 

\item[Random Generation] Random Generation, has 
interesting qualities, but is slow to create interesting worlds and it best
suited to fixed sizes. It is possible to use an overall general map, but 
difficult to reproduce out of ordered generations which we are likely to
experience. 

\item[Perlin Noise] Noise based generation is fast, at least compared to the other
methods, and is entirely reproducible from any state. These two qualities make it
perfect for our requirements. Noise can also be used on multiple dimensions, 
allowing height maps to be generated as well as 3D density maps leading to very
interesting realistic worlds.
\end{description}

Considering our requirements noised based generation is the best overall 
approach,
but as more complex worlds are create inspiration will come from other 
techniques.\\
