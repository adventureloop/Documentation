Computer games create an escape for players from the the norms of their
lives into worlds where anything is possible. They give access to endless stories
from hundreds of different narrators and place the player into worlds with nearly
endless possibilities.\\

The scene of one of these adventures can be anywhere, from the confines of a 
space ship on trading runs across the galaxy, riding a horse through the planes
of the old west or fighting tooth and nail into a castle to sack it to the 
ground. Despite the setting the determining success of computer games is the
immersive and 
expansive world that the player can explore. \\

Large, featureful worlds are by their nature difficult, time consuming and 
expensive to create. Even the modest setting of a first person shooter can take
many man years to produce and the placement of every tree in a forest is a very
demanding task. To alleviate the costs of creating interesting worlds for a 
player to navigate many developers and game studios use procedural techniques to
add a layer of realism. They may specify the general features of a region and 
fill these gaps in with generated content.\\

The final realisation of this idea is to leave the creation of the game world 
entirely up to generation algorithms. This technique allows world of infinite size
or the creation of entire Galaxys. Perfecting algorithms for every case can be 
difficult, but highly realistic worlds can be created without the need for 
massive data stores. If generation is taken as a core game mechanic, the nature
of the game will change from its advertised functionality to using the setting 
more and more. If done well players will pass around the worlds they have been
given to share the experience of something particularly unique.\\
