\section{Project Goals}
\subsection{Rendering System}
Before generation can be useful a system is required that can be used to 
visualise the final product. The first goal of the project is to create a
rendering system capable of displaying a very large generated world efficiently 
which takes advantage of modern graphics techniques.\\

\subsection{Random Terrain Generation}
The defining feature of a generated world is the structure of the terrain, this
goal will encompass the use of random based algorithms to generated a 
topologically interesting world.\\

\subsection{Generation Techniques}
A featureless landscape can be interesting to view, but for most games more is 
required to create an immersive world. Generating of interesting natural features
such as tree lines, rivers and man made features such as structures and villages
allow the creation of such worlds that will hold a players attention.

\subsection{Entity System}
The final piece that makes a world interesting to explore are other actors to 
interact with. The addition of an entity system and extensions to the generator
to place these, allow worlds that can give the player bodies to encounter.\\

\section{Motivation}
I came across the game nethack in 2006 through the webcomic user friendly, over time
I encounter all of the games mention in this chapter as well as many others that used
generation not just to create their world,but as a core game mechanic. I was 
introduced to voxel worlds through the game minecraft and I wanted to explore
dynamic level creation in games since. 

