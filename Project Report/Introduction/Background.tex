Generation has been used in games for decades, it was originally taken advantage
of to allow reduction in the amount of disk space required to store the game.
With modern games generation is used at all levels, from building the world the
user operates within, to allowing dynamic events and quests for the player to
complete. Most modern games involve generation in some form, lack of such now 
will cause games to be called static, unimaginative or on rails.\\

Generated worlds are at the extreme end of the scale, but since the beginning of
the games industry generated worlds have been a valuable technique for developers
and players alike.\\

\subsection{Nethack, A Decade spanning game}
In 1984 nethack began development under the aupicous of what would become a 
shadowy core team. In game play it followed in the lineage of the genre defining
rogue hack and slash game. Now decades later nethack still draws new players,
in the first case to its lengendary difficulty, but secondly due to nethacks 
randomly generated dungeons.\\ 

In nethack the players character enters the dungeons of doom, vertically stacked
series of dungeons that hold somewhere around the 26th level the Amulet of 
Yendor. The players goal is to desecend into the dungeons of doom, find the
amulet and return back to the surface. This is no simple task, the dungeons are 
different on every play through and contain a multitude of evil monsters trying
to halt your progres. Nethack is famous for it taking years for players to 
complete their first asscention.\\

Nethack, unlike its predecessors generates the dungeons in the majority of levels
differently on each play through. This technique makes in near impossible to 
provide a run through guide and instead encourages players to develop winning
strategies.\\

%Include images

Nethack has captured generations of players and developers, resulting in many 
user written interfaces and tools to aid exploration or to put a friendlier 
interface on the game. Nethack has also created a cult following within nerd 
culture, best described by the User Friendly web comic.\\

\subsection{Minecraft, A modern blockbuster success}
In 2005 Marus 'Notch' Peerson, a developer for an online cassino released an
early version of a game concept he had been working on in 'online forum'. This
game that later became the blockbust Minecraft show cases a world made of cubes 
or voxels, this world was generated for the player from either a random or 
player specified seed. The talking point for this game was the interactivity of
the world, any and all of the voxel elements in the world are changable by the
player. With this feature minecraft became a digital lego set and quickly 
reached cult success.\\

% Minecraft lego

Minecrafts generated worlds have become one of the most interesting features of 
the game. A sub culture exists where players create near photo realistic renders
of vistas from the game, some even go to the extent of having 3d models of their
creations made. 

%beautiful minecraft.


 
