\section{Background}
There is a long history of generated worlds being used as a core mechanic, 
originally used to get around space limitations with early computers generation
has moved to a core mechanic in some games or as a helpful tool in others. Prime
examples of the use of generation as a setting and as core game play follow.\\

\subsection*{Dwarf Fortress, The Forever Game}
Dwarf Fortress is described as the ultimate simulation game and after 8 years of
development is only near a 0.2 release. The developer Tarns Adams plans to create
a realistic simulation of a fantasy world and has created a small cult following
around his project. Everything in Dwarf Fortress is generated a long simulation
rules, with the initial world the player builds in being simulated from planet 
formation. Everything in the world is generated, including history's, religions
other fortresses and world events. \\


\begin{figure}[h!]
  \includegraphics[width=\textwidth]{Images/DwarfFortress.png}
  \caption{Dwarf Fortress Embark View}
\end{figure}

Dwarf Fortress creates a simulated world that the player can play in multiple 
times and it is awesome.\\

\subsection*{Nethack, A Decade spanning game}
In 1984 Nethack began development under the aupicous of what would become a 
shadowy core team. In game play it followed in the lineage of the genre defining
rogue hack and slash game. Now decades later Nethack still draws new players,
in the first case to its legendary difficulty, but secondly due to nethacks 
randomly generated dungeons.\\ 

\begin{figure}[h!]
  \includegraphics[width=\textwidth]{Images/Nethack.png}
  \caption{Nethack}
\end{figure}

In Nethack the players character enters the dungeons of doom, vertically stacked
series of dungeons that hold somewhere around the 26th level the Amulet of 
Yendor. The players goal is to descend into the dungeons of doom, find the
amulet and return back to the surface. This is no simple task, the dungeons are 
different on every play through and contain a multitude of evil monsters trying
to halt your progress. Nethack is famous for it taking years for players to 
complete their first Ascension.\\

Nethack, unlike its predecessors generates the dungeons in the majority of levels
differently on each play through. This technique makes in near impossible to 
provide a run through guide and instead encourages players to develop winning
strategies.\\

Nethack has captured generations of players and developers, resulting in many 
user written interfaces and tools to aid exploration or to put a friendlier 
interface on the game. Nethack has also created a cult following within nerd 
culture, best described by the User Friendly web comic.\\

\subsection*{Minecraft, A modern blockbuster success}
In 2005 Marus 'Notch' Peerson, a developer for an on line casino released an
early version of a game concept he had been working on in 'on line forum'. This
game that later became the blockbuster Minecraft show cases a world made of cubes 
or voxels, this world was generated for the player from either a random or 
player specified seed. The talking point for this game was the interactivity of
the world, any and all of the voxel elements in the world are changeable by the
player. With this feature Minecraft became a digital Lego set and quickly 
reached cult success.\\

\begin{figure}[h!]
  \includegraphics[width=\textwidth]{Images/Minecraft.png}
  \caption{Minecraft Sunset}
\end{figure}

Minecrafts generated worlds have become one of the most interesting features of 
the game. A sub culture exists where players create near photo realistic renders
of vistas from the game, some even go to the extent of having 3d models of their
creations made. \\
