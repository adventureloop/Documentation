\section{Background}
There is a long history of generated worlds being used as a core mechanic, 
originally used to get around space limitations with early computers generation
has moved to a core mechanic in some games or as a helpful tool in others. Prime
examples of the use of generation as a setting and as core game play follow.\\

\subsection*{Dwarf Fortress, The Forever Game}
Dwarft Fortress \cite{DwarfFortress} is described as a single player 
fantasy simulation game,
based in a 
randomly generated persistent world. The description is very simple, but the game is
famous for its very steep learning curve and prohibitive UI. Dwarf Fotress is 
probably the perfect example of a fan driven fan, with a cult following despite the
initial difficultly in playing the game.\\

Dwarf Fortess uses simulated world to their maxium extent, with the land features 
being generate, but also full histories that include prominent characters and 
significant world events. In addition to this generation worlds can be revisited by
players in different runs through the game giving it true depth.\\

\begin{figure}[h!]
  \includegraphics[width=\textwidth]{Images/DwarfFortress.png}
  \caption{Dwarf Fortress Embark View}
\end{figure}

For world generation DF uses a combination of random fractal terrain and simulated
world effects such a erosion, weather patterns and tectonic movements.\\

\subsection*{Nethack, A Decade spanning game}
In 1984 Nethack \cite{Nethack} began development under the aupicous of what would 
become a 
shadowy core team. In game play it followed in the lineage of the genre defining
rogue hack and slash game. Now decades later Nethack still draws new players,
in the first case due to its legendary difficulty, but secondly due to nethacks 
randomly generated dungeons.\\ 

\begin{figure}[h!]
  \includegraphics[width=\textwidth]{Images/Nethack.png}
  \caption{Nethack}
\end{figure}

In Nethack the players character enters the dungeons of doom, vertically stacked
series of dungeons that hold somewhere around the 26th level the Amulet of 
Yendor. The players goal is to descend into the dungeons of doom, find the
amulet and return back to the surface. This is no simple task, the dungeons are 
different on every play through and contain a multitude of obstacles
trying to halt your progress. Nethack is famous for it taking years for players to 
n
complete their first Ascension.\\

Nethack, unlike its predecessors generates the dungeons in the majority of levels
differently on each play through. This technique makes in near impossible to 
provide a run through guide and instead encourages players to develop winning
strategies.\\

Nethack has captured generations of players and developers, resulting in many 
user written interfaces and tools to aid exploration or to put a friendlier 
interface on the game. Nethack has also created a cult following within nerd 
culture, best described by the User Friendly web comic.\\

\subsection*{Minecraft, A modern blockbuster success}
In 2005 Marus 'Notch' Peerson, a developer for an online casino released an
early version of a game concept he had been working on in 'online forum'. This
game that later became the blockbuster Minecraft \cite{Minecraft} show cases a 
world made of cubes 
or voxels, this world was generated for the player from either a random or 
player specified seed. The talking point for this game was the interactivity of
the world, any and all of the voxel elements in the world are changeable by the
player. With this feature Minecraft became a digital Lego set and quickly 
reached cult success.\\

\begin{figure}[h!]
  \includegraphics[width=\textwidth]{Images/Minecraft.png}
  \caption{Minecraft Sunset}
\end{figure}

Minecrafts generated worlds have become one of the most interesting features of 
the game. A sub culture exists where players create near photo realistic renders
of vistas from the game, some even go to the extent of having 3d models of their
creations made. \\
